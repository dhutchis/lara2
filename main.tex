% This is "sig-alternate.tex" V2.1 April 2013
% This file should be compiled with V2.5 of "sig-alternate.cls" May 2012
%
% This example file demonstrates the use of the 'sig-alternate.cls'
% V2.5 LaTeX2e document class file. It is for those submitting
% articles to ACM Conference Proceedings WHO DO NOT WISH TO
% STRICTLY ADHERE TO THE SIGS (PUBS-BOARD-ENDORSED) STYLE.
% The 'sig-alternate.cls' file will produce a similar-looking,
% albeit, 'tighter' paper resulting in, invariably, fewer pages.
%
% ----------------------------------------------------------------------------------------------------------------
% This .tex file (and associated .cls V2.5) produces:
%       1) The Permission Statement
%       2) The Conference (location) Info information
%       3) The Copyright Line with ACM data
%       4) NO page numbers
%
% as against the acm_proc_article-sp.cls file which
% DOES NOT produce 1) thru' 3) above.
%
% Using 'sig-alternate.cls' you have control, however, from within
% the source .tex file, over both the CopyrightYear
% (defaulted to 200X) and the ACM Copyright Data
% (defaulted to X-XXXXX-XX-X/XX/XX).
% e.g.
% \CopyrightYear{2007} will cause 2007 to appear in the copyright line.
% \crdata{0-12345-67-8/90/12} will cause 0-12345-67-8/90/12 to appear in the copyright line.
%
% ---------------------------------------------------------------------------------------------------------------
% This .tex source is an example which *does* use
% the .bib file (from which the .bbl file % is produced).
% REMEMBER HOWEVER: After having produced the .bbl file,
% and prior to final submission, you *NEED* to 'insert'
% your .bbl file into your source .tex file so as to provide
% ONE 'self-contained' source file.
%
% ================= IF YOU HAVE QUESTIONS =======================
% Questions regarding the SIGS styles, SIGS policies and
% procedures, Conferences etc. should be sent to
% Adrienne Griscti (griscti@acm.org)
%
% Technical questions _only_ to
% Gerald Murray (murray@hq.acm.org)
% ===============================================================
%
% For tracking purposes - this is V2.0 - May 2012

%\documentclass{sig-alternate-05-2015}
\documentclass{sig-alternate} %[conference]{IEEEtran}

\setlength{\paperheight}{11in}

\newcommand{\lara}{\textsc{Lara}}%
\newcommand{\laradb}{\textsc{LaraDB}}%
\newcommand{\plara}{\textsc{PLara}}%
\newcommand{\matlab}{MATLAB}%
\usepackage{ mathdots }

\usepackage[dvipsnames,table]{xcolor}
\usepackage{bm}
\usepackage{amsmath}
% \usepackage{amsfonts}
\usepackage{amssymb}
% \let\proof\undefined
% \usepackage{amsthm}
\usepackage{amscd}
\usepackage{mathpartir}

\usepackage{subfig}
\usepackage{adjustbox}
\usepackage{calc}
\usepackage{color}
\usepackage{varwidth}
\usepackage{multirow}
% \usepackage{xfrac}

% \theoremstyle{definition}
% \newtheorem{definition}{Definition}[section]

\newlength{\centersep}
\setlength{\centersep}{-.5em}

\usepackage{relsize}
\newcommand{\cmark}{\ding{51}}%
\newcommand{\xmark}{\ding{55}}%

\newcommand{\newcomment}[3]{%
  \expandafter\newcommand\csname #1\endcsname[1]{{\color{#2}{[#3: ##1]}}}%
}
\newcomment{bill}{blue}{Bill}

% outer joins
\def\ojoin{\setbox0=\hbox{$\bowtie$}%
  \rule[-.02ex]{.25em}{.4pt}\llap{\rule[\ht0]{.25em}{.4pt}}}
\def\leftouterjoin{\mathbin{\ojoin\mkern-5.8mu\bowtie}}
\def\rightouterjoin{\mathbin{\bowtie\mkern-5.8mu\ojoin}}
\def\fullouterjoin{\mathbin{\ojoin\mkern-5.8mu\bowtie\mkern-5.8mu\ojoin}}

\usepackage{ wasysym }
\newcommand{\mxz}{{\,\oplus.\oland\,}}
\newcommand{\cxm}{{\,\omid.\otimes\,}}
\newcommand{\oxm}[1][]{{\,\fullmoon_{#1}.\otimes\,}}
% \newcommand{\join}{\bowtie}
% \newcommand{\sjoin}{\hat{\join}}
\newcommand\uproduct{%
  \mathchoice{\mathbin{\;\rotatebox{90}{$\Bowtie$}}}%
             {\mathbin{\;\rotatebox{90}{$\Bowtie$}}}%
             {\mathbin{\;\rotatebox{90}{\scalebox{0.65}{$\Bowtie$}}}}%
             {\mathbin{\;\rotatebox{90}{\scalebox{0.65}{$\Bowtie$}}}}%
}

\newcommand{\collen}{238}
\newcommand{\bif}{\text{ if }}
\newcommand{\bthen}{\text{ then }}
\newcommand{\belse}{\text{ else }}
\newcommand{\botherwise}{\text{ otherwise }}
\newcommand{\temperature}{temp}
\newcommand{\humidity}{hum}
\newcommand{\myfighsep}{$\quad$}
\newcommand{\overbar}[1]{\mkern 1.5mu\overline{\mkern-1.5mu#1\mkern-1.5mu}\mkern 1.5mu}
\newcommand{\tup}[1]{\overbar{#1}}
\newcommand{\ex}[0]{\mathbb{E}}
\newcommand{\bone}[0]{\mathbf{1}}
\newcommand{\bzero}[0]{\mathbf{0}}
\newcommand{\bbot}{} % 

\newcommand{\bb}[1]{\textsc{#1}}
\newcommand{\bsort}[0]{\textbf{S{\scriptsize ORT}}}
\newcommand{\bext}{\bb{Ext}}
\newcommand{\bmap}{\bb{Map}}
\newcommand{\bby}{\bb{by}}
\newcommand{\bon}{\bb{on}}
\newcommand{\bto}{\bb{to}}
\newcommand{\bfrom}{\bb{from}}
\newcommand{\bjoin}{\bb{Join}}
\newcommand{\bunion}{\bb{Union}}
\newcommand{\bagg}{\bb{Agg}}
\newcommand{\bselect}{\bb{select}}
\newcommand{\bas}{\bb{as}}
\newcommand{\bgroupby}{\bb{group by}}
\newcommand{\brename}{\bb{Rename}}
\newcommand{\bmergejoin}{\bb{MergeJoin}}
\newcommand{\bmergeunion}{\bb{MergeUnion}}
\newcommand{\bmergeagg}{\bb{MergeAgg}}
\newcommand{\bsortagg}{\textbf{S{\scriptsize ORT}A{\scriptsize GG}}}
\newcommand{\bapply}{\bb{Map}}
\newcommand{\bapplyext}{\bb{Ext}}
\newcommand{\bwhere}{\bb{where}}
\newcommand{\bload}{\bb{Load}}
\newcommand{\bstore}{\bb{Store}}
\newcommand{\bover}{\bb{over}}
\newcommand{\bnull}{\bb{null}}

\usepackage{xparse}
\DeclareDocumentCommand{\join}{ O{} O{} }{\operatorname{\bowtie_{#1}^{#2}}\!}
\newcommand{\joino}[1][]{\join[\otimes][#1]}
\newcommand{\joinp}[1][]{\join[\otimes][#1]} %\odot
\newcommand{\joinotup}[1][]{\join[\tup{\otimes}][#1]}
\DeclareDocumentCommand{\sjoin}{ O{} O{} }{\operatorname{\hat{\bowtie}_{#1^{#2}}}}
\newcommand{\sjoino}[1][]{\sjoin[\otimes][#1]}
\newcommand{\sjoinotup}[1][]{\sjoin[\tup{\otimes}][#1]}
\DeclareDocumentCommand{\union}{ O{} O{} }{\operatorname{\uproduct_{#1}^{#2}}\!}
\newcommand{\uniono}[1][]{\union[\oplus][#1]} %\operatorname{\uproduct_\oplus^{#1}}
\newcommand{\unionp}[1][]{\union[+][#1]} %\operatorname{\uproduct_\oplus^{#1}}
\newcommand{\unionotup}[1][]{\union[\tup{\oplus}][#1]} %\operatorname{\uproduct_\tup{\oplus}^{#1}}
\newcommand{\tupotimes}[1][]{\operatorname{\tup{\otimes}}^{#1}}
\newcommand{\tupoplus}[1][]{\operatorname{\tup{\oplus}}^{#1}}
\DeclareMathOperator{\ext}{ext}
\DeclareMathOperator{\avg}{avg}
\DeclareMathOperator{\coarsen}{coarsen}
\DeclareMathOperator{\reduce}{reduce}
\DeclareMathOperator{\map}{map}
\DeclareMathOperator{\supp}{supp}
\DeclareMathOperator{\trace}{tr}
\DeclareMathOperator*{\tupbigoplus}{\tup{\bigoplus}}
\DeclareMathOperator*{\tupbigotimes}{\tup{\bigotimes}}

\usepackage{mathtools}
\DeclarePairedDelimiter{\ceil}{\lceil}{\rceil}
\DeclarePairedDelimiter{\floor}{\lfloor}{\rfloor}
\DeclarePairedDelimiter{\paren}{(}{)}

\newcommand{\lrparen}[1]{% \llrrparen{..}
  \left(#1\right)}
\newcommand{\lrbracket}[1]{% \lrbracket{..}
  \left[#1\right]}
\newcommand{\lrceil}[1]{% \lrbracket{..}
  \left\lceil#1\right\rceil}
\newcommand{\llrrbracket}[1]{% \llrrbracket{..}
  \left[\mkern-3mu\left[#1\right]\mkern-3mu\right]}
\newcommand{\llrrparen}[1]{% \llrrparen{..}
  \left(\mkern-6mu\left(#1\right)\mkern-6mu\right)}

\newcommand{\dylan}[1]{\textcolor{teal}{{\bf DH:} #1}}
\newcommand{\tr}[0]{{\intercal}}
\newcommand{\col}[0]{\colon\!}
\newcommand{\matr}[1]{\mathbf{#1}} % undergraduate algebra version
\newcommand{\mxm}{{\,\oplus.\otimes\,}}
% to allow [H] for algorithms
% https://tex.stackexchange.com/questions/82271/multiple-algorithm2e-algorithms-in-two-column-documents/82272#82272
\makeatletter
\newcommand{\removelatexerror}{\let\@latex@error\@gobble}
\makeatother
% vspace above and below inline algorithms
\newlength{\algspace}
\setlength{\algspace}{3pt}
\usepackage[]{algorithm2e}


\usepackage[  
  bookmarks=false,
  hyperfootnotes=false,
  hyperindex=false,
  %pdfpagemode=UseNone,
  %bookmarksopen=false,
  hidelinks,   
%  pagebackref=true,
    pdftitle={New Lara: A Minimalist Kernel for Linear and Relational Algebra Computation},  
  pdfauthor={Dylan Hutchison, Bill Howe, Dan Suciu},
  pdfpagelabels=false]{hyperref}


\def\sharedaffiliation{%
\end{tabular}
\begin{tabular}{c}}


%% new command for indent for algorithm.
\newlength\myindent
\setlength\myindent{2em}
\newcommand\bindent{%
  \begingroup
  \setlength{\itemindent}{\myindent}
  \addtolength{\algorithmicindent}{\myindent}
}
\newcommand\eindent{\endgroup}

\usepackage{todonotes}
\newcommand{\todoi}[1]{\todo[inline]{#1}}

\graphicspath{{.}{./figures/}}

%\usepackage{dblfloatfix} %provides: \usepackage{fixltx2e}
\usepackage{siunitx}
\sisetup{round-precision=2,round-mode=places,scientific-notation=true}
\usepackage{threeparttable}
\usepackage{multirow}
\usepackage{adjustbox}
\usepackage{array}
\newcolumntype{R}[2]{%
    >{\adjustbox{angle=#1,lap=\width-(#2)}\bgroup}%
    l%
    <{\egroup}%
}

% biblatex redefines citeyear
% make sure the ACM version is used
% \let\citeyearold\citeyear
% \let\citeyear\undefined 
% \usepackage[style=trad-abbrv,backend=bibtex]{biblatex} % multiple .bib files
% \addbibresource{header.bib}
% \addbibresource{refs.bib}
% \let\citeyear\citeyearold
% \let\citeyearold\undefined



\usepackage{listings}










\begin{document}

% Copyright
% \setcopyright{acmcopyright}
%\setcopyright{acmlicensed}
\setcopyright{rightsretained}
%\setcopyright{usgov}
%\setcopyright{usgovmixed}
%\setcopyright{cagov}
%\setcopyright{cagovmixed}


% DOI
% \doi{10.475/123_4}

% ISBN
% \isbn{123-4567-24-567/08/06}

%Conference
% \conferenceinfo{EDBT}{March 21--24, 2017, Venice, Italy}
% \acmPrice{\$15.00}

%
% --- Author Metadata here ---
%\CopyrightYear{2007} % Allows default copyright year (20XX) to be over-ridden - IF NEED BE.
%\crdata{0-12345-67-8/90/01}  % Allows default copyright data (0-89791-88-6/97/05) to be over-ridden - IF NEED BE.
% --- End of Author Metadata ---


\title{New Lara: A Minimalist Kernel for \\ Linear and Relational Algebra Computation}
% \subtitle{[Extended Abstract]
% \titlenote{A full version of this paper is available as
% \textit{Author's Guide to Preparing ACM SIG Proceedings Using
% \LaTeX$2_\epsilon$\ and BibTeX} at
% \texttt{www.acm.org/eaddress.htm}}}

\numberofauthors{1}
\author{
\alignauthor
	Dylan Hutchison, Bill Howe, Dan Suciu\\[3pt]
     \affaddr{\{dhutchis,billhowe,suciu\}@cs.washington.edu}%
       % \affaddr{Department of Computer Science and Engineering}\\
       % \affaddr{University of Washington}\\
       % \affaddr{Seattle, WA 98195-2350, U.S.A.}
% \begin{tabular}{c}
% %\alignauthor
% 	David Maier\\
% 	\affaddr{maier@cs.washington.edu}\\
%        \affaddr{Department of Computer Science}\\
%        \affaddr{Portland State University}\\
%        \affaddr{Portland, OR  97207-0751, U.S.A.}
% \end{tabular}
% 
% \vspace{-20em}%
}

% RA and LA need to be used together
% 

\maketitle
\begin{abstract}
\lara{} is good! \lara{} is great!
\end{abstract}
% Some point about how this is more explicit than MapReduce (and, dare we say, lambda calculus)
% while less explicit than either RA or LA. It's at just the right level of expressiveness.


% NoSQL key-value databases emphasize record-level read-write operations, 
% relying on external infrastructure such as Spark or MapReduce 
% to implement complex analytics (e.g., matrix math, relational joins, machine learning). 
% Computing in an external system is expensive for small yet complex analytics, 
% requiring long code paths to extract data from the database and prepare native data structures.
% In response, developers implement custom applications 
% that push simple filters and sums into the database's scans
% in order to maximize in-database processing.
% Recent software generalized this approach 
% to provide native, in-database support for complex analytics.

% In this work we evaluate the performance of in-database vs. external system 
% approaches to query processing for the Apache Accumulo NoSQL database.
% Specifically we run Graphulo, a library for matrix math 
% inside Accumulo's scan-time iterators, and MapReduce, 
% an off-the-shelf external system commonly used with Accumulo,
% on sparse matrix multiplication.
% Results indicate that the Graphulo's in-database approach is superior at smaller problem sizes,
% while at larger problem sizes the two approaches have similar performance.
\section{Introduction}


\begin{figure*}[t]
% \begin{tabular}[t]{cc|c}
%  % & & [$f(k_1,k_2)$] \\
% $k_1$ & $k_2$ & $v$ \\
% \hline
% $a$ & $x$ & 2 \\
% $c$ & $x$ & 3 \\
% -- & -- & 0
% \end{tabular}
% $\;\join[+]\;$
% \begin{tabular}[t]{c|c}
% $k_1$ & $v$ \\
% \hline
% $a$ & 7 \\
% $b$ & 1 \\
% $c$ & 5 \\
% --  & 0
% \end{tabular}
% $\;=\;$
% \begin{tabular}[t]{cc|c}
% $k_1$ & $k_2$ & $v$ \\
% \hline
% $a$ & $x$ & 9 \\
% $a$ & -- & 7 \\
% $b$ & -- & 2 \\
% $c$ & $x$ & 8 \\
% $c$ & -- & 3 \\
% -- & -- & 0
% \end{tabular}
% \\

% \begin{tabular}[t]{cc|c}
% $k_1$ & $k_2$ & $v$ \\
% \hline
% $a$ & $x$ & 9 \\
% $a$ & -- & 7 \\
% $b$ & -- & 2 \\
% $c$ & $x$ & 8 \\
% $c$ & -- & 3 \\
% -- & -- & 0
% \end{tabular}
% $\;\union[+]\;$
% \begin{tabular}[t]{c|c}
% $k_1$ & $v$ \\
% \hline
% -- & 0
% \end{tabular}
% $\;=\;$
% \begin{tabular}[t]{c|r}
% $k_1$ & $v$ \\
% \hline
% $a$ & 9 + $\lim 7 + \lim 0 = 9 + \infty + 0 = \infty$ \\
% $b$ & $\lim 2 + \lim 0 = \infty + 0 = \infty$ \\
% $c$ & 8 + $\lim 3 + \lim 0 = 8 + \infty + 0 = \infty$ \\
% -- & $\lim 0 = 0$ \\
% \end{tabular}
% \\

% \begin{tabular}[t]{cc|l}
% $k_1$ & $k_2$ & $v$ \\
% \hline
% $a$ & $x$ & 9 \\
% $a$ & -- & 7 \\
% $b$ & -- & 2 \\
% $c$ & $x$ & 8 \\
% $c$ & -- & 3 \\
% -- & -- & 0
% \end{tabular}
% $\;\union[+]\;$
% \begin{tabular}[t]{c|c}
% $k_2$ & $v$ \\
% \hline
% -- & 0
% \end{tabular}
% $\;=\;$
% \begin{tabular}[t]{c|r}
% $k_2$ & $v$ \\
% \hline
% $x$ & 9 + $8 + 2 + \lim 0 + \lim 0 = 19 + 0 + 0 = 19$ \\
% -- & $7 + 2 + 3 + \lim 0 + \lim 0 = 12 + 0 + 0 = 12$ \\
% \end{tabular}

% \[ \lim x = \begin{cases} \infty, &x>0 \\ -\infty, &x<0 \\ 0, &x=0 \end{cases} \]

% \begin{align*}
% A \join[\odot] B &= \ext_{\text{\begin{tabular}{c|c}$k_B$ & $v$ \\ \hline -- & $v$\end{tabular}}} (A) 
% \odot \ext_{\text{\begin{tabular}{c|c}$k_A$ & $v$ \\ \hline -- & $v$\end{tabular}}} (B) \\
% A \union[\odot] B &= \gamma^{k_A \cap k_B}_{\odot} (A) \odot \gamma^{k_A \cap k_B}_{\odot} (B)
% \end{align*}

Core operations:
\begin{enumerate}\itemsep0pt
\item \bext{} $A$ \bby{} $f$
\item \bagg{} $A$ [\bon{} $\tup{k}$] \bby{} $\oplus$ \textsc{limit} $lim$
\item \textsc{Combine} $A, B, \dots$ \bby{} $\odot$
\end{enumerate}
Derived operations:
\begin{enumerate}\itemsep0pt
\item \bmap{} $A$ \bby{} $f$
\item \bext{} $A$ \textsc{with} $\tup{k}$ = \bext{} $A$ \bby{} \begin{tabular}{c|c}$\tup{k}$ & $\tup{v}$ \\ \hline -- & $\tup{v}$\end{tabular} (add blank keys $\tup{k}$)
\item \brename{} $A$ \bfrom{} $x$ \bto{} $y$ = $\dots$
\item \bjoin{} $A, B$ \bby{} $\odot$ = \textsc{Combine} (\bext{} $A$ \textsc{with} $\tup{k}_B$), (\bext{} $B$ \textsc{with} $\tup{k}_A$) \bby{} $\odot$
\end{enumerate}
Physical \lara{}:
\begin{enumerate}\itemsep0pt
\item \bsort{} $A$ \bon{} $\tup{k}$
\item \bagg{} $A$ [\bon{} $\tup{k}$] \bby{} $\oplus$ \textsc{limit} $lim$
\item \bsortagg{} $A$ \bon{} $\tup{k}$ \bby{} $\oplus$ \textsc{limit} $lim$
\item \textsc{Merge} $A,B,\dots$ \bby{} $\odot$
\end{enumerate}


\begin{tabular}{c|cccc}
  &   & x & y &   \\
\hline
a & 0 & 2 & 3 & 0 \\
b & 0 & 4 & 0 & 0 \\
  & 0 & 0 & 0 & 0
\end{tabular}
+
10 =
\begin{tabular}{c|cccc}
  &   & x & y &   \\
\hline
a & 10 & 12 & 13 & 10 \\
b & 10 & 14 & 10 & 10 \\
  & 10 & 10 & 10 & 10
\end{tabular}

\begin{tabular}{c|cccc}
  &   & x & y &   \\
\hline
a & 0 & 2 & 3 & 0 \\
b & 0 & 4 & 0 & 0 \\
  & 0 & 0 & 0 & 0
\end{tabular}
+
\begin{tabular}{c|cccc}
  &   & x & y &   \\
\hline
a & 10 & 10 & 10 & 10 \\
b & 10 & 10 & 10 & 10 \\
  & 10 & 10 & 10 & 10
\end{tabular}
=
\begin{tabular}{c|cccc}
  &   & x & y &   \\
\hline
a & 10 & 12 & 13 & 10 \\
b & 10 & 14 & 10 & 10 \\
  & 10 & 10 & 10 & 10
\end{tabular}

\begin{tabular}{c|cccc}
  &   & x & y &   \\
\hline
a & 0 & 2 & 3 & 0 \\
b & 0 & 4 & 0 & 0 \\
  & 0 & 0 & 0 & 0 \\
  & 0 & 0 & 0 & 0 \\
\end{tabular}
+
\begin{tabular}{c|c}
  &   \\
a & 10 \\
  & 0  \\
c & 12 \\
  & 0  \\
\end{tabular}
=
\begin{tabular}{c|cccc}
  &   & x & y &   \\
\hline
a & 10 & 12 & 13 & 10 \\
b & 0 & 4 & 0 & 0 \\
c & 12 & 12 & 12 & 12 \\
 & 0 & 0 & 0 & 0 \\
\end{tabular}

\begin{tabular}{c|cccc}
  &   & x & y &   \\
\hline
a & 0 & 2 & 3 & 0 \\
b & 0 & 4 & 0 & 0 \\
  & 0 & 0 & 0 & 0 \\
  & 0 & 0 & 0 & 0 \\
\end{tabular}
+
\begin{tabular}{c|cccc}
  &   &  &  &   \\
\hline
a & 10 & 10 & 10 & 10 \\
  & 0 & 0 & 0 & 0 \\
c & 12 & 12 & 12 & 12 \\
  & 0 & 0 & 0 & 0 \\
\end{tabular}
=
\begin{tabular}{c|cccc}
  &   & x & y &   \\
\hline
a & 10 & 12 & 13 & 10 \\
b & 0 & 4 & 0 & 0 \\
c & 12 & 12 & 12 & 12 \\
 & 0 & 0 & 0 & 0 \\
\end{tabular}


$\forall k, k',\; f(k,0)(k') = 0'$ 

% \begin{tabular}{c|ccccc}
% &$\dots$&$\dots$&$x$&$\dots$&$\dots$\\
% \hline
% $\vdots$ & $\ddots$ & $\vdots$ & $\vdots$ & $\vdots$ & $\iddots$ \\
% $\vdots$ & $\dots$ & 2 & 2 & 2 &  $\dots$ \\
% $a$ & $\dots$ & 2 & 5 & 2 &  $\dots$ \\
% $\vdots$ & $\dots$ & 2 & 2 & 2 &  $\dots$ \\
% $\vdots$ & $\iddots$ & $\vdots$ & $\vdots$ & $\vdots$ & $\ddots$ \\
% \end{tabular}
\end{figure*}

\section*{Acknowledgments}
{\footnotesize This material is based upon work supported by the NSF Graduate
Research Fellowship under Grant No. DGE-1256082.}
% Opinions, findings, and
% conclusions or recommendations expressed in this material are those of the author and do not
% necessarily reflect the views of the National Science Foundation.

%
% The following two commands are all you need in the
% initial runs of your .tex file to
% produce the bibliography for the citations in your paper. 
% \renewcommand*{\bibfont}{\footnotesize}
\nocite{hutchison2016lara}
\bibliographystyle{abbrv}
{\small 
\bibliography{header,refs}
}
%\bibliographystyle{abbrv}
%\printbibliography{}
%
% ACM needs 'a single self-contained file'!
%
%\balancecolumns % GM June 2007
\end{document}
